% !TEX encoding = UTF-8 Unicode 
% !TEX root = praca.tex

\chapter{Języki regularne}
W rozdziale tym omówimy krótko pojęcie wyrażenia regularnego, języka regularnego, automatu skończonego oraz związków między tymi pojęciami. Wszystkie pojęcia oraz wyniki omawiane w tym rozdziale znajdują się w pierwszych rozdziałach klasycznej książki \cite{HopUll2003}. 

Wyrażenia regularne nad alfabetem $\Sigma$ definiowane są za pomocą następujących rekurencyjnych reguł:
\begin{enumerate}
    \item $\emptyset$, $\varepsilon$ (słowo puste), $a$ (dla dowolnego $a\in\Sigma$) są wyrażeniami regularnymi,
    \item jeśli $r_1$ oraz $r_2$ są wyrażeniami regularnymi, to
    \begin{itemize}
        \item[$\bullet$] $r_1+r_2$ (suma)
        \item[$\bullet$] $r_1 r_2$ (konkatenacja)
        \item[$\bullet$] $r_1^*$ (domknięcie Kleene'ego)
        \item[$\bullet$] $(r_1)$ (grupowanie)
    \end{itemize}
    są wyrażeniami regularnymi.
\end{enumerate}
Z każdym wyrażeniem regularnym $r$ związany jest język $L(r)$ reprezentowany przez $r$, który definiowany jest za pomocą następujących reguł:
\begin{enumerate}
    \item $L(\emptyset) = \emptyset$
    \item $L(\varepsilon) = \{\varepsilon\}$
    \item $L(a) = \{a\}$ dla $a \in \Sigma$
    \item $L(r_1+r_2) = L(r_1)\cup L(r_2)$
    \item $L(r_1 r_2) = \{xy:x \in L(r_1) \land y\in L(r_2)\}$, gdzie $xy$ oznacza konkatenację ciągów $x$ oraz $y$
    \item $L(r^*) = L(r)^*$, gdzie $L^*$ definiujemy następująco:\\
    $\begin{aligned}[t]
        &L^0=\{\varepsilon\}\\
        &L^{n+1}=\{vw: v\in L^n\wedge w\in L\}\\
        &L^*=\bigcup_{n=0}^{\infty}{L^n}
    \end{aligned}$
\end{enumerate}
przy czym gwiazdka wiąże najsilniej, konkatenacja słabiej, a suma najsłabiej (przykładowo nie możemy zastosować reguły $L(aa+bb)=\{xy: x\in L(a) \wedge y\in L(a+bb)\}$, gdyż konkatenacja wiąże silniej niż suma, natomiast zachodzi $L(aa+bb)=L(aa)\cup L(bb)$). Języki postaci $L(r)$ dla wyrażania regularnego nazywamy językami regularnymi.

\section{Języki regularne i automaty skończone}

Najczęstszym problemem dotyczącym wyrażeń regularnych jest sprawdzanie, czy dane słowo należy języka generowanego przez to wyrażenie oraz wyszukiwanie słów z tego języka w tekście. W celu zautomatyzowania tego procesu konstruuje się automaty skończone.
\begin{definition}
    \emph{Deterministycznym automatem skończonym (DFA)} nazywamy strukturę $(\Sigma, Q, q_0, \delta, F)$, gdzie
    \begin{enumerate}
        \item $\Sigma$ jest skończonym zbiorem symboli nazywanym alfabetem,
        \item $Q$ jest skończonym zbiorem stanów automatu,
        \item $q_0 \in Q$ jest stanem początkowym automatu,
        \item $\delta:Q\times \Sigma\to Q$ jest funkcją przejścia, która parze $(q,a)$ przypisuje stan, w jakim znajdzie się automat po przeczytaniu symbolu $a$ w stanie $q$
        \item $F$ jest zbiorem stanów akceptujących (końcowych).
    \end{enumerate}
\end{definition}
Funkcja $\delta$ może być również określona częściowo. Wówczas z niektórych stanów $q$ nie ma przejścia do kolejnego stanu po przeczytaniu pewnego symbolu $a$, dla którego wartość $\delta(q,a)$ nie jest określona. Mówimy, że słowo $w=a_1\dotsm a_n$ jest akceptowalne przez deterministyczny automat skończony $M=(Q,\Sigma,q_0,\delta,F)$, jeśli istnieją takie stany $q_1,\dotsc,q_n\in Q$, że $\delta(q_{k-1},a_k)=q_k$ dla $k=1,\dotsc,n$ oraz stan $q_n$ jest akceptowalny.
\begin{definition}
    \emph{Niedeterministycznym automatem skończonym (NFA)} nazywamy piątkę $(\Sigma, Q, q_0, \delta, F)$, gdzie
    \begin{enumerate}
        \item $\Sigma$ jest skończonym alfabetem,
        \item $Q$ jest skończonym zbiorem stanów automatu,
        \item $q_0 \in Q$ jest stanem początkowym automatu,
        \item $\delta:Q\times \Sigma\to P(Q)$ jest funkcją przejścia, która parze $(q,a)$ przypisuje zbiór stanów, w jakich może znaleźć się automat po przeczytaniu symbolu $a$ w stanie $q$ ($P(Q)$ jest zbiorem potęgowym zbioru $Q$),
        \item $F$ jest zbiorem stanów akceptujących.
    \end{enumerate}
\end{definition}
\begin{definition}
    \emph{Niedeterministyczny automat skończony z $\varepsilon$-przejściami ($\varepsilon$-NFA)} definiujemy analogicznie do NFA bez $\varepsilon$-przejść z tą różnicą, że dodajemy możliwość przejścia z jednego stanu automatu do drugiego po słowie pustym, czyli $\delta:Q\times(\{\varepsilon\}\cup\Sigma)\to P(Q)$. 
\end{definition}
Mówimy, że słowo $w=a_1\dotsm a_n$ jest akceptowalne przez niedeterministyczny automat skończony $M=(Q,\Sigma,q_0,\delta,F)$, jeśli istnieją takie stany $q_1,\dotsc,q_n\in Q$, że $q_k\in\delta(q_{k-1},a_k)$ dla $k=1,\dotsc,n$ oraz stan $q_n$ jest akceptowalny.

Mówimy, że automat skończony $M=(Q,\Sigma,q_0,\delta,F)$ (deterministyczny lub niedeterministyczny) akceptuje język $L$, jeśli $L$ jest zbiorem słów akceptowalnych przez $M$. Język akceptowany przez $M$ oznaczamy $L(M)$.

Przedstawimy teraz kilka twierdzeń dotyczących wyrażeń regularnych i automatów skończonych.

\begin{theorem}
    Niech $r$ będzie wyrażeniem regularnym. Wtedy istnieje $\varepsilon$-NFA akceptujący język $L(r)$.
\end{theorem}

\begin{theorem}
    Niech $L$ będzie zbiorem akceptowalnym przez $\varepsilon$-NFA. Wtedy istnieje NFA bez $
    \varepsilon$-przejść akceptujący $L$.    
\end{theorem}

\begin{theorem}
    Niech $L$ będzie zbiorem akceptowalnym przez NFA. Wtedy istnieje DFA akceptujący $L$.
\end{theorem}

\begin{theorem}
    Niech $L$ będzie zbiorem akceptowalnym przez DFA. Wtedy $L$ jest reprezentowany przez wyrażenie regularne.
\end{theorem}

Z powyższych twierdzeń wynika, że wszystkie przedstawione tu pojęcia są równoważne. W szczególności dla każdego wyrażenia regularnego możemy stworzyć deterministyczny automat skończony akceptujący język generowany przez to wyrażenie. 


Omówmy teraz pokrótce metodę konstrukcji DFA na podstawie danego wyrażenia regularnego. Jest ona podzielona na kilka kroków odpowiadających powyżej wymienionym twierdzeniom.

W pierwszym kroku konstruujemy $\varepsilon$-NFA odpowiadający danemu wyrażeniu regularnemu według następujących reguł:
\begin{enumerate}
    \item Dla wyrażenia $\emptyset$ tworzymy automat składający się z jednego stanu nieakceptującego, a wyrażenie $\varepsilon$ odpowiada automatowi złożonemu z jednego stanu akceptującego. Język generowany przez wyrażenie $a$ dla $a\in\Sigma$ jest akceptowalny przez automat, który zawiera stan początkowy, który nie jest akceptujący oraz drugi stan, który jest akceptujący, przy czym istnieje jedno przejście ze stanu początkowego do drugiego stanu po symbolu $a$.
    \item Dla wyrażenia regularnego $r=r_1+r_2$ i automatów $M_1=(Q_1, \Sigma_1, q_1, \delta_1, F_1)$ i $M_2=(Q_2, \Sigma_2, q_2, \delta_2, F_2)$ takich, że $L(M_1)=L(r_1)$ i $L(M_2)=L(r_2)$, automat $M$ akceptujący język $L(r)$ tworzymy w ten sposób, że tworzymy nowy stan nieakceptowalny $q_0$ i łączymy ten stan ze stanami początkowymi automatów $M_1$ i $M_2$ $\varepsilon$-przejściami. Otrzymujemy w ten sposób automat $$M=(Q_1\cup Q_2 \cup \{q_0\}, \Sigma_1\cup\Sigma_2, q_0, \delta_1\cup\delta_2\cup\{((q_0,\varepsilon),\{q_1,q_2\})\}, F_1\cup F_2)~.$$
    \item W przypadku, gdy $r=r_1 r_2$, automat $M$ akceptujący język $L(r)$ tworzymy w ten sposób, że każdy stan akceptujący automatu $M_1$ łączymy $\varepsilon$-przejściem ze stanem akceptującym automatu $M_2$. Ponadto wszystkie stany akceptujące automatu $M_1$ zamieniamy na stany nieakceptujące w nowym automacie $M$, a stanem początkowym automatu $M$ jest stan początkowy automatu $M_2$. Otrzymujemy w ten sposób automat
    $$M=(Q_1\cup Q_2,\Sigma_1\cup\Sigma_2,q_1,\delta,F_2)~,$$
    gdzie
    \begin{itemize}
        \item[$\bullet$] $\delta(q,a)=\delta_1(q,a)$ dla $q\in Q_1\setminus F_1$ i $a\in\Sigma_1\cup\{\varepsilon\}$,
        \item[$\bullet$] $\delta(q,a)=\delta_1(q,a)\cup\{q_2\}$ dla $q\in Q_1\cap F_1$ i $a\in\Sigma_1\cup\{\varepsilon\}$,
        \item[$\bullet$] $\delta(q,a)=\delta_2(q,a)$ dla $q\in Q_2$ i $a\in\Sigma_2\cup\{\varepsilon\}$.
    \end{itemize}
    \item W przypadku, gdy $r=r_1^*$, automat $M$ akceptujący język $L(r)$ tworzymy z automatu $M_1$ w ten sposób, że każdy stan akceptujący automatu $M_2$ łączymy $\varepsilon$-przejściem ze stanem początkowym tego automatu oraz ustawiamy ten stan początkowy na stan akceptujący. W ten sposób otrzymujemy automat
    $$M=(Q_1,\Sigma_1,q_1,\delta,F_1\cup\{q_1\})~,$$
    gdzie
    \begin{itemize}
        \item[$\bullet$] $\delta(q,a)=\delta_1(q,a)$ dla $q\in Q_1\setminus F_1$ i $a\in\Sigma_1\cup\{\varepsilon\}$,
        \item[$\bullet$] $\delta(q,a)=\delta_1(q,a)\cup\{q_1\}$ dla $q\in Q_1\cap F_1$ i $a\in\Sigma_1\cup\{\varepsilon\}$.
    \end{itemize}
\end{enumerate}
Usuwanie epsilon przejść jest nieco bardziej skomplikowanym procesem. Niech $M_0=(Q_0,\Sigma_0,q_0,\delta_0,F_0)$ będzie niedeterministycznym automatem skończonym z $\varepsilon$ przejściami. W pierwszym kroku, w celu ograniczenia liczby stanów, możemy wyznaczyć silnie spójne składowe podgrafu zawierającego tylko $\varepsilon$-przejścia, a następnie złączyć ze sobą wszystkie stany należące do jednej silnie spójnej składowej, gdyż w takiej składowej z każdego stanu jesteśmy w stanie przejść do innego stanu wyłącznie po $\varepsilon$-przejściach. Niech $V_1,\dotsc,V_n$ będą takimi silnie spójnymi składowymi w automacie $M_0$ i niech $q_k'$ będzie stanem, na który zamieniamy składową $V_k$ dla $k=1,\dots,n$. Niech $M_k=(Q_k,\Sigma_k,q_k,\delta_k,F_k)$ będzie automatem po zamianie składowych $V_1,\dotsc,V_k$ na pojedyncze stany. Automat $M_k$ otrzymujemy w następujący sposób:
\begin{enumerate}
    \item $Q_k=Q_{k-1}\cup\{q_k'\}\setminus V_k$
    \item $\Sigma_k=\Sigma_{k-1}$
    \item $q_k = \left\{
        \begin{array}{rl}
            q_{k-1}, & q_{k-1}\notin V_k\\
            q_k', & q_{k-1}\in V_k
        \end{array}
        \right.$
    \item Na początek każde przejście z dowolnego stanu z $V_k$ chcemy zamienić na przejście ze stanu $q_k'$. Następnie każde przejście z dowolnego stanu do stanu z $V_k$ chcemy zamienić na przejście do stanu $q_k'$, za wyjątkiem $\varepsilon$-przejść między dwoma stanami z $V_k$ (wówczas stworzylibyśmy $\varepsilon$-przejście z $q_k'$ do $q_k'$, które jest niepotrzebne). Zdefiniujmy pomocniczo
    $$\left\{
        \begin{array}{rcll}
            \delta_{k-1}(q_k',a)&=&\bigcup_{q\in V_k}{\delta_{k-1}(q,a)}, & a\in\Sigma_k\\
            \delta_{k-1}(q_k',\varepsilon)&=&\left(\bigcup_{q\in V_k}{\delta_{k-1}(q,\varepsilon)}\right)\setminus V_k
        \end{array}
        \right.~.$$
    Dla $q\in Q_k$ i $a\in\Sigma_k\cup\varepsilon$ definiujemy
    $$\delta_k(q,a)=\left\{
        \begin{array}{rl}
            \delta_{k-1}(q,a), & \delta_{k-1}(q,a)\cap V_k=\emptyset\\
            \delta_{k-1}(q,a)\cup\{q_k'\}\setminus V_k, & \delta_{k-1}(q,a)\cap V_k\neq\emptyset
        \end{array}
        \right.~.$$
    \item $F_k = \left\{
        \begin{array}{rl}
            F_{k-1}, & F_{k-1}\cap V_k=\emptyset\\
            F_{k-1}\cup\{q_k'\}\setminus V_k, & F_{k-1}\cap V_k\neq\emptyset
        \end{array}
        \right.$
\end{enumerate}
Po tym kroku w automacie $M_n$ nie istnieją cykle złożone z samych $\varepsilon$-przejść (inaczej stany należące do takiego cyklu należałyby do jednej silnie spójnej składowej). W tym momencie wyznaczamy $\varepsilon$-domknięcia każdego stanu, czyli zbiór stanów, do których możemy dotrzeć z danego stanu po $\varepsilon$-przejściach. Dla $q\in Q_n$ $\varepsilon$-domknięcia możemy wyznaczyć rekurencyjnie w następujący sposób
$$\varepsilon\text{-DOMKN}(q)=\left\{
        \begin{array}{rl}
            q & \delta_n(q,\varepsilon)\subseteq\{q\}\\
            q\cup\bigcup_{p\in\delta_n(q,\varepsilon)}{\varepsilon\text{-DOMKN}(p)}, & \delta_n(q,\varepsilon)\nsubseteq\{q\}
        \end{array}
        \right.~.$$
Dzięki temu, że w automacie $M_n$ nie ma cykli złożonych z samych $\varepsilon$-przejść, przy obliczaniu $\varepsilon$-domknięć kolejnych stanów unikniemy zapętleń. Spróbujmy teraz NFA bez $\varepsilon$-przejść $\hat{M}=(\hat{Q},\hat{\Sigma},\hat{q_0},\hat{\delta},\hat{F})$. Dla $P\subseteq \hat{Q}=Q_n$ zdefiniujmy $\varepsilon\text{-DOMKN}(P)=\bigcup_{q\in P}{\varepsilon\text{-DOMKN}(q)}$. Dla $q\in \hat{Q}$ i $w\in\hat{\Sigma}=\Sigma_n$ niech $\hat{\delta}(q,w)$ oznacza zbiór stanów, w których możemy się znaleźć po przeczytaniu słowa $w$ w stanie $q$. Wówczas
$$\left\{
\begin{array}{rcll}
    \hat{\delta}(q,\varepsilon)&=&\varepsilon\text{-DOMKN}(q)\\
    \hat{\delta}(q,wa)&=&\varepsilon\text{-DOMKN}\left(\bigcup_{p\in\hat{\delta}(q,w)}{\delta_n(p,a)}\right), & w\in\Sigma^*,a\in\Sigma
\end{array}
\right.~.$$
W celu zbudowania NFA bez $\varepsilon$-przejść musimy wyznaczyć $\hat{\delta}(q,a)$ dla wszystkich $q\in \hat{Q}$ i $a\in\hat{\Sigma}$. Korzystając z powyższego wzoru otrzymujemy
$$\hat{\delta}(q,a)=\varepsilon\text{-DOMKN}\left(\bigcup_{p\in\varepsilon\text{-DOMKN}(q)}{\delta_n(p,a)}\right)~.$$
Elementem początkowym w automacie $\hat{M}$ jest oczywiście element początkowy w automacie $M_n$, czyli $\hat{q_0}=q_n$. Pozostało nam wyznaczenie stanów akceptowalnych w automacie $\hat{M}$. Do każdego stanu, do którego mogliśmy dotrzeć w automacie $M_n$ po pewnym niepustym słowie, jesteśmy w stanie dotrzeć również w automacie $\hat{M}$. Jedynymi stanami osiągalnymi w automacie $M_n$, do których nie jesteśmy w stanie dotrzeć w automacie $\hat{M}$, są te stany należące do $\varepsilon$-domknięcia stanu początkowego $\hat{q_0}$, do których w automacie $M_n$ możemy się dostać jedynie po słowie pustym, za wyjątkiem stanu początkowego $\hat{q_0}$. W związku z tym, jeśli któryś z tych stanów jest akceptowalny, a stan początkowy nie jest akceptowalny w automacie $\hat{M}$, to automat $\hat{M}$ nie akceptowałby słowa pustego, a powinien je akceptować. Stąd stan początkowy $\hat{q_0}$ jest akceptujący w automacie $\hat{M}$ wtedy i tylko wtedy, gdy istnieje stan akceptujący należący do $\varepsilon\text{-DOMKN}(\hat{q_0})$. Ponadto, jeśli nie zmienimy akceptowalności pozostałych stanów, to $\hat{M}$ będzie akceptował te same słowa, które akceptuje automat $M_n$, gdyż do każdego stanu akceptującego, do którego możemy dojść po niepustym słowie w automacie $M_n$, możemy również dojść w automacie $\hat{M}$. Stąd
$$\hat{F}=\left\{
\begin{array}{rl}
    F_n, & F_n\cap\varepsilon\text{-DOMKN}(q_n)=\emptyset\\
    F_n\cup\{q_n\}, & F_n\cap\varepsilon\text{-DOMKN}(q_n)\neq\emptyset
\end{array}
\right.~.$$
Wiemy już, jak zbudować NFA na podstawie wyrażenia regularnego. Ostatnim krokiem, jaki musimy podjąć, jest zamiana NFA na DFA. Niech $M=(Q,\Sigma,q_0,\delta,F)$ będzie niedeterministycznym automatem skończonym bez $\varepsilon$-przejść. Deterministyczny automat skończony $M'=(Q',\Sigma',q_0',\delta',F')$ budujemy w następujący sposób:
\begin{enumerate}
    \item $\Sigma'=\Sigma$, $q_0'=\{q_0\}$ jest stanem początkowym automatu $M'$.
    \item Zbiór stanów $Q'$ i funkcję przejścia $\delta'$ wyznaczamy jak poniżej.
    \begin{enumerate}
        \item Początkowym zbiorem stanów jest $Q_0'=\{q_0'\}=\{\{q_0\}\}$. Podstawiamy $k=0$ i dla kolejnych $k$ wykonujemy poniższy krok, dopóki $Q_{k+1}'\neq Q_k'$.
        \item Zdefiniujmy pomocniczo $Q_{-1}'=\emptyset$. Niech $P_k$ oznacza zbiór nieodwiedzonych stanów, czyli $P_k=Q_k'\setminus Q_{k-1}'$. Dla każdego stanu $q'\in P_k$ i każdego symbolu $a\in\Sigma$ wyznaczamy $\delta'(q',a)=\bigcup_{q\in q'}{\delta(q,a)}$. Jeśli $\delta'(q',a)$ nie należy do $Q_k'$, tworzymy nowy stan $p=\delta'(q',a)$ i dodajemy go do zbioru $P_{k+1}$. Otrzymujemy
        $$P_{k+1}=\left\{\bigcup_{q\in q'}{\delta(q,a)}: q'\in Q_k'\right\}\setminus Q_k'$$
        oraz $Q_{k+1}'=Q_k'\cup P_{k+1}$.
        \item Niech $n$ będzie największą liczbą $k$, dla której wykonujemy powyższy krok. Finalnym zbiorem stanów jest zbiór $Q'=Q_n'$.
    \end{enumerate}
    \item $F'=\{q'\in Q': F\cap q'\neq\emptyset\}$.
\end{enumerate}
Niech $w=a_1\dotsm a_m$ będzie słowem akceptowalnym przez $M$. Pokażemy, że słowo $w$ jest akceptowalne w $M'$. Niech $q_0,q_1,\dotsc,q_m$ będą kolejnymi stanami, w których znajdziemy się odczytując słowo $w$, takimi, że $q_m\in F$. Niech $q_k'=\delta'(q_{k-1}',a_k)$ dla $k\geq 1$. Mamy $q_0\in q_0'$ oraz przy założeniu, że $q_{k-1}\in q_{k-1}'$, zachodzi
$$q_k\in\delta(q_{k-1},a_k)\subseteq\bigcup_{q\in q_{k-1}'}{\delta(q,a_k)}=\delta'(q_{k-1}',a_k)=q_k'~.$$
Wobec tego $q_k\in q_k'$ dla $k=1,\dotsc,m$, w szczególności $q_m\in q_m'$, co w połączeniu z tym, że $q_m\in F$, daje $q_m'\cap F\neq\emptyset$, czyli $q_m'$ jest stanem akceptowalnym, zatem słowo $w$ jest akceptowalne w $M'$.\\
Przeprowadźmy dowód w drugą stronę. Niech $w=a_1\dotsm a_m$ będzie słowem akceptowalnym przez $M'$. Niech $q_k'=\delta'(q_{k-1}',a_k)$ dla $k=1,\dotsc,m$. Stan $q_m'$ jest akceptowalny, zatem musi istnieć taki stan $q_m\in F$, że $q_m\in q_m'$. Dla $k=1,\dotsc,m$ mamy
$$q_k'=\delta'(q_{k-1}',a_k)=\bigcup_{q\in q_{k-1}'}{\delta(q,a_k)}~,$$
zatem dla każdego stanu $p\in q_k'$ istnieje stan $q\in q_{k-1}'$ taki, że $p\in\delta(q,a_k)$. Dla $k=m,m-1,\dotsc,1$ niech $q_{k-1}\in\{q_{k-1}'\}$ będzie dowolnym takim stanem, że $q_k\in\delta(q_k,a_k)$. Oznaczenie to nie jest sprzeczne z tym, że $q_0$ jest elementem początkowym automatu $M$, gdyż jest to jedyny element należący do $q_0'$. Wówczas $q_0,q_1,\cdots,q_m$ jest jedną z możliwych ścieżek, które możemy obrać przechodząc po kolejnych symbolach słowa $w$ w automacie $M$. Ponadto stan $q_m$ jest akceptowalny, zatem słowo $w$ jest akceptowalne w automacie $M$. Wobec powyższych wniosków automaty $M$ i $M'$ są równoważne.

Zauważmy, że $F'\subseteq P(Q)$. W najgorszym przypadku przy zamianie NFA na DFA następuje więc wykładnicza eksplozja liczby stanów. W celu osiągnięcia jak najmniejszej możliwej liczby stanów automatu, automat ten możemy poddać minimalizacji. Algorytm minimalizacji deterministycznego automatu skończonego $M=(Q,\Sigma,q_0,\delta,F)$ przebiega następująco:
\begin{enumerate}
    \item Będziemy oznaczać te pary stanów, które nie są równoważne i których nie możemy złączyć w jeden stan.
    \item Dla każdych dwóch stanów $p\in F$ oraz $q\in Q\setminus F$ oznaczamy pary $(p,q)$ i $(q,p)$.
    \item Oznaczmy przez $L_q$ język słów, które byłyby akceptowane przez $M$, gdyby stanem początkowym tego automatu był stan $q$. Dwa stany $p$ i $q$ są równoważne wtedy i tylko wtedy, gdy $L_p=L_q$. Jeżeli istnieje słowo $w\in\Sigma^*$ takie, że po przeczytaniu słowa $w$ w stanie $p$ znajdziemy się w stanie akceptowalnym, a po przeczytaniu tego słowa w stanie $q$ znajdziemy się w stanie nieakceptowalnym (lub na odwrót), to stany $p$ i $q$ nie są równoważne. Dla każdej pary różnych stanów $(p,q)\in (F\times F)\cup(Q\setminus F\times Q\setminus F)$ sprawdzamy, czy istnieje symbol $a\in\Sigma$ taki, że para $(\delta(p,a),\delta(q,a))$ jest oznaczona.
    \begin{enumerate}
        \item Jeżeli nie istnieje taki symbol, oznacza to, że stany $p$ i $q$ są równoważne pod warunkiem, że żadna z par $(\delta(p,a),\delta(q,a))$ nie zostanie w przyszłości oznaczona. Jeśli bowiem para $(\delta(p,b),\delta(q,b))$ zostanie oznaczona dla pewnego $b\in\Sigma$, to istnieje słowo $w$ takie, że po przeczytaniu słowa $w$ w stanach $\delta(p,b)$ i $\delta(q,b)$ znajdziemy się w dwóch różnych stanach, z których jeden jest akceptowalny, a drugi nie. Po przeczytaniu słowa $bw$ w stanach $p$ i $q$ znaleźlibyśmy się w tych samych stanach, co po przeczytaniu słowa $w$ w stanach $\delta(p,b)$ i $\delta(q,b)$, zatem stany $p$ i $q$ nie są równoważne. W celu oznaczenia pary $(p,q)$ w momencie, w którym oznaczamy pewną parę $(\delta(p,b),\delta(q,b))$, a para $(p,q)$ była odwiedzona wcześniej, dla każdej pary stanów $(q_1,q_2)$ będziemy tworzyć listę par $(p_1,p_2)$ takich, że istnieje symbol $a\in\Sigma$ taki, że $\delta(p_1,a)=q_1$ oraz $\delta(p_2,a)=q_2$, a para $(p_1,p_2)$ była nieoznaczona w momencie dodawania jej do listy związanej z parą $(q_1,q_2)$. W tym momencie dla każdego symbolu $a\in\Sigma$ do listy $(\delta(p,a),\delta(q,a))$ dodajemy parę $(p,q)$, o ile $\delta(p,a)\neq\delta(q,a)$.
        \item W przeciwnym razie istnieje symbol $a\in\Sigma$, dla którego para $(\delta(p,a),\delta(q,a))$ jest oznaczona i musimy oznaczyć parę $(p,q)$. W tym momencie musimy również oznaczyć wszystkie pary $(p',q')$ na liście $(p,q)$, następnie wszystkie pary na listach $(p',q')$ i tak dalej, zatem oznaczamy wszystkie te pary z kolejnych list w sposób rekurencyjny.
    \end{enumerate}
    Po wykonaniu powyższych kroków dla wszystkich par $(p,q)\in (F\times F)\cup(Q\setminus F\times Q\setminus F)$, wszystkie pary nierównoważnych stanów są oznaczone, a pozostałe pary stanów są równoważne. Niech $P_1,\dotsc,P_n$ będą zbiorami równoważnych stanów. Dla każdego z tych zbiorów wybierzmy po jednym reprezentancie $p_i\in P_i$. Nowym zbiorem stanów będzie zbiór $Q'=\{p_i:i\in[n]\}$. Niech $j(q)$ dla $q\in Q$ oznacza taką liczbę, że $q\in P_{j(q)}$. Niech $M'=(Q',\Sigma,q_0',\delta',F')$ będzie minimalnym DFA otrzymanym z $M$. $M'$ otrzymujemy w ten sposób, że $q_0'=p_{j(q_0)}$, $F'=F\cap Q'$ oraz dla każdego $i\in[n]$ i każdego symbolu $a\in\Sigma$ mamy $\delta'(p_i,a)=p_{j(\delta(p_i,a))}$.
    
\end{enumerate}