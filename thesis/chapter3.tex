\chapter{Problem zliczania słów}

W rozdziale tym będziemy się zajmowali następującym zagadnieniem: mamy ustalony język regularny $R$ nad skończonym alfabetem $\Sigma$; dla $n \in \mathbb{N}$ definiujemy 
$$r_n = card( \{x\in R: |x| = n\})~;$$ 
chcemy wyznaczyć asymptotykę ciągu $(r_n)_{n\geq 0}$. 
W wielu sytuacjach możliwe będzie wyznaczenie zwartych formuł na elementy $r_n$, z których asymptotyka będzie łatwa do wyznaczenia.  

Naszym głównym narzędziem będą funkcje tworzące związane z rozważanym językiem $R$. Są to szeregi potęgowe postaci 
$$R(x) = \sum_{n=0}^{\infty} r_n x^n~.$$
Funkcję tę będziemy nazywali funkcją tworzącą języka $R$.

\begin{lemma} Jeśli $R$ jest językiem regularnym nad skończonym alfabetem $\Sigma$, to promień zbieżności $r$ funkcji tworzącej $R(x)$ spełnia nierówność
$$ r \geq \frac{1}{card(\Sigma)}~.$$
\end{lemma}
\begin{proof}
    Z kryterium Cauchy'ego zbieżności szeregów  wiemy, że promień zbieżności szeregu potęgowego $R(x)$ jest równy
    $$r = \frac{1}{\limsup_{n \to \infty}{\sqrt[n]{|r_n|}}}~.$$
    Ponadto, korzystając z definicji liczb $r_n$, otrzymujemy
    $$r_n = card(\{x\in R : |x| = n\}) \leq card(\Sigma^{[n]}) = card(\Sigma)^n~.$$
    Stąd
    $$r = \frac{1}{\limsup_{n \to \infty}{\sqrt[n]{|r_n|}}} \geq \frac{1}{\limsup_{n \to \infty}{\sqrt[n]{card(\Sigma)^n}}} = \frac{1}{card(\Sigma)}~.$$
\end{proof}
Z udowodnionego Lematu wynika, że funkcja tworząca języka regularnego nad skończonym alfabetem jest analityczna w pewnym otoczeniu zera. Do badania jej własności można więc stosować metody analizy matematycznej oraz zespolonej.

\section{Jednoznaczność i wieloznaczność wyrażeń}

Rozważania w tym rozdziale rozpoczniemy od rozważenia pewnego przykładu.
Niech alfabet $\Sigma = \{a,b\}$. Wyrażenie regularne
\begin{equation} \label{eq:reg1}
r = (a+b)^* aa (a+b)^*
\end{equation}
generuje język $L(r)$ złożony z tych skończonych ciągów $x \in \Sigma^*$, w których występuje podciąg dwóch kolejnych wystąpień litery $a$. Bezpośrednie przetłumaczenie tego wyrażenia regularnego na język struktur kombinatorycznych prowadzi do struktury
$$
  \mathcal{C} = \mathbf{SEQ}(\{a,b\}) \times \{a\} \times \{a\} \times \mathbf{SEQ}(\{a,b\})~,
$$
której funkcja tworząca wyraża się wzorem
$$
  \mathcal{C}(x) = \frac{1}{1-2x} \cdot x \cdot x \cdot \frac{1}{1-2x} = \frac{x^2}{(1-2x)^2}~,
$$
a jej współczynniki wynoszą
$$
  [x^n]\mathcal{C}(x) = (n-1)2^{n-2}
$$
dla $n>0$. 

W każdym słowie $x \in L(r)$ przed pierwszym wystąpieniem podciągu $aa$ znajduje się pewien ciąg symboli $a$ i $b$ (możliwe, że pusty), w którym nie występuje podciąg $aa$, natomiast po pierwszym wystąpieniu podciągu $aa$ w słowie $x$ znajduje się dowolny ciąg znaków $a$ i $b$. Język $L(r)$ możemy więc również opisać za pomocą wyrażenia regularnego
\begin{equation} \label{eq:reg2}
    r' = b^*(abb^*)^*aa(a+b)^*~.
\end{equation}
Ponadto, bezpośrednie przetłumaczenie wyrażenia $r'$ na język struktur kombinatorycznych generuje klasę
$$
  \mathcal{C'} = \mathbf{SEQ}(\{b\})\times\mathbf{SEQ}(\{a\}\times\{b\}\times\mathbf{SEQ}(\{b\}))\times\{a\}\times\{a\}\times\mathbf{SEQ}(\{a,b\})~,
$$
której funkcja tworząca wyraża się równaniem
$$
  \mathcal{C'}(x) = \frac{1}{1-x}\cdot \frac{1}{1-\frac{x^2}{1-x}} \cdot x \cdot x \cdot \frac{1}{1-2x} = \frac{x^2}{(1-x-x^2)(1-2x)} = \frac{1}{1-2x}-\frac{1+x}{1-x-x^2}~,
$$
z czego po przekształceniach wnioskujemy, że
\begin{equation}\label{eq:correctFormula}
  [x^n]\mathcal{C'}(x) = 2^n - F_{n+2}~,
\end{equation}
gdzie $F_k$ oznacza $k$-tą liczbę Fibonacciego. Zauważmy, że tłumaczenie wyrażeń $\ref{eq:reg1}$ i $\ref{eq:reg2}$ dało dwa różne wyniki, mimo, że $L(r)=L(r')$, z czego wynika, że \textbf{naiwne tłumaczenie wyrażeń regularnych na język struktur kombinatorycznych nie zawsze daje prawidłową funkcję tworzącą języka}. 
Warto w tym miejscu zaznaczyć, że to wzór (\ref{eq:correctFormula}) poprawnie zlicza rozważany język, co pokażemy w części \ref{chapt:wybrane}.

\subsection{Wyrażenia jednoznaczne}

W celu wyjaśnienia różnicy między rozważanymi wyrażeniami wprowadźmy pojęcie jednoznaczności wyrażenia regularnego.

Wprowadźmy oznaczenie $r^n=\underbrace{r\dotsm r}_n$ ($r^0=\varepsilon$), gdzie $r$ jest wyrażeniem regularnym. Mówimy, że wyrażenie regularne jest jednoznaczne, jeśli każde słowo należące do języka generowanego przez to wyrażenie możemy dopasować do tego wyrażenia tylko na jeden sposób (\cite{Flajolet_08} strona 734). Równoważnie, wyrażenie regularne $r$ nad alfabetem $\Sigma$ jest jednoznaczne, gdy spełnia jeden z następujących warunków:
\begin{enumerate}
    \item $r=\emptyset \vee r=\varepsilon \vee r=a$, gdzie $a\in\Sigma$,
    \item $r=r_1+r_2$, gdzie wyrażenia $r_1$ i $r_2$ są jednoznaczne i języki $L(r_1)$ i $L(r_2)$ są rozłączne
    \item $r=r_1r_2$, gdzie wyrażenia $r_1$ i $r_2$ są jednoznaczne oraz dla każdego słowa $x\in L(r)$ istnieje dokładnie jedna para słów $y$ i $z$ takich, że $y\in L(r_1)$ i $z\in L(r_2)$ oraz $yz=x$
    \item $r=r_1^*$, gdzie dla każdego $n$ wyrażenie
    $$\varepsilon+r_1+r_1^2+\ldots+r_1^n$$
    jest jednoznaczne (równoważnie dla każdych różnych $m$ i $n$ wyrażenie $r_1^n$ jest jednoznaczne, a języki $L(r_1)^n=L(r_1^n)$ i $L(r_1)^m=L(r_1^m)$ są rozłączne).
\end{enumerate}
Należy przy tym pamiętać, że gwiazdka wiąże najsilniej, konkatenacja słabiej, a suma najsłabiej.

Wróćmy do przykładu z wyrażeniami \ref{eq:reg1} i \ref{eq:reg2}. Wyrażenie $r=(a+b)^*aa(a+b)^*$ nie jest jednoznaczne, gdyż słowo $x=aabaa$ możemy wyprowadzić na dwa sposoby: $x=y_1 y_2 y_3$, gdzie $y_1=aab\in L((a+b)^*)$, $y_2=aa\in L(aa)$, $y_3=\varepsilon\in L((a+b)^*)$ lub $x=z_1 z_2 z_3$, gdzie $z_1=\varepsilon \in L((a+b)^*)$, $z_2=aa\in L(aa)$, $z_3=baa\in L((a+b)^*)$.\\
Przyjrzyjmy się wyrażeniu $r'=b^*(abb^*)^*aa(a+b)^*$. Jasne jest, że wyrażenia $ab$ i $aa$ są jednoznaczne. Wyrażenia $b^n$ są jednoznaczne oraz języki $L(b^n)$ i $L(b^m)$ są rozłączne dla różnych $n$ i $m$, więc $b^*$ jest wyrażeniem jednoznacznym. Podobnie wyrażenia $(a+b)^n$ są jednoznaczne (można to udowodnić indukcyjnie), a języki $L((a+b)^n)$ i $L((a+b)^m)$ są rozłączne dla różnych $n$ i $m$, gdyż wszystkie słowa z tych języków mają długości odpowiednio $n$ i $m$, zatem wyrażenie $(a+b)^*$ jest jednoznaczne. Dla każdego $n$ wyrażenie $(abb^*)^n$ jest jednoznaczne, gdyż wyrażenie $abb^*$ jest jednoznaczne, a każde słowo $x$ należące do języka $L((abb^*)^{n+1})$ możemy jednoznacznie przedstawić jako $x=yz$, gdzie $y\in L((abb^*)^n)$, a $z\in L(abb^*)$ jest sufiksem słowa $x$ zaczynającym się od ostatniego symbolu $a$ w słowie $x$ (dowód indukcyjny po $n$). Ponadto każde słowo z języka $L((abb^*)^n)$ zawiera dokładnie $n$ symboli $a$, więc języki $L((abb^*)^n)$ i $L((abb^*)^m)$ są rozłączne dla $n\neq m$, zatem wyrażenie $(abb^*)^*$ jest jednoznaczne. Idąc dalej, nie jest trudnym udowodnić, że wyrażenia $b^*(abb^*)^*$ i $aa(a+b)^*$ są jednoznaczne. Wobec tego każde słowo $x\in L(r')$ możemy jednoznacznie przedstawić w formie $x=yz$, gdzie $y\in L(b^*(abb^*)^*)$, a $z\in L(aa(a+b)^*)$ jest sufiksem słowa $x$, który zaczyna się od pierwszego wystąpienia podciągu $aa$ w słowie $x$, zatem finalnie wyrażenie $r'$ jest jednoznaczne.

\begin{theorem}
  Niech $r$ będzie jednoznacznym wyrażeniem regularnym. Wówczas funkcję tworzącą języka generowanego przez wyrażenie $r$ możemy obliczyć według następujących reguł:
  \begin{enumerate}
      \item $L(\emptyset)(x)=0$, $L(\varepsilon)(x)=1$, $L(a)(x)=x$ dla $a\in\Sigma$,
      \item jeśli $r=r_1+r_2$, to $L(r)(x)=L(r_1)(x)+L(r_2)(x)$,
      \item jeśli $r=r_1 r_2$, to $L(r)(x)=L(r_1)(x)\cdot L(r_2)(x)$,
      \item jeśli $r=r_1^*$, to $L(r)(x)=\frac{1}{1-L(r_1)(x)}$.
  \end{enumerate}
\end{theorem}
\begin{proof}

Przypadek 1. Równości $L(\emptyset)(x)=0$, $L(\varepsilon)(x)=1$ i $L(a)(x)=x$ dla $a\in\Sigma$ są oczywiste.

\noindent\\%
Przypadek 2. Jeśli $r=r_1+r_2$ i $r$ jest wyrażeniem jednoznacznym, to języki $L(r_1)$ i $L(r_2)$ są rozłączne, zatem
        $$\begin{aligned}[t]
            L(r)(x)&=L(r_1+r_2)(x)=(L(r_1)\cup L(r_2))(x)\\
            &=(L(r_1)+L(r_2))(x)=L(r_1)(x)+L(r_2)(x)~.
        \end{aligned}$$
Przypadek 3. Jeśli $r=r_1 r_2$ i $r$ jest wyrażeniem jednoznacznym, to dla każdego słowa $t\in L(r)$ istnieje dokładnie jedna para słów $(y,z)\in L(r_1)\times L(r_2)$ taka, że $t=yz$, zatem
        $$\begin{aligned}[t]
            L(r)(x)&=L(r_1 r_2)(x)=\{yz:y\in L(r_1)\wedge z\in L(r_2)\}(x)\\
            &=\sum_{y\in L(r_1)}{\sum_{z\in L(r_2)}{\frac{x^{|yz|}}{|\{(y',z')\in L(r_1)\times L(r_2):y'z'=yz\}|}}}\\
            &=\sum_{y\in L(r_1)}{\sum_{z\in L(r_2)}{x^{|y|+|z|}}}=\left(\sum_{y\in L(r_1)}{x^{|y|}}\right)\left(\sum_{z\in L(r_2)}{x^{|z|}}\right)\\
            &=L(r_1)(x)\cdot L(r_2)(x).
        \end{aligned}$$
Przypadek 4. Niech $r=r_1^*$. Korzystając z jednoznaczności wyrażenia $r$ otrzymujemy
        $$\begin{aligned}[t]
            L(r)(x)&=L(r_1^*)(x)=L(r_1)^*(x)=\left(\bigcup_{n\in\mathbb{N}}{L(r_1)^n}\right)(x)=\left(\sum_{n\in\mathbb{N}}{L(r_1)^n}\right)(x)\\
            &=(\varepsilon+L(r_1)+L(r_1)\times L(r_1)+L(r_1)\times L(r_1) \times L(r_1)+\ldots)(x)\\
            &=\mathbf{SEQ}(L(r_1))(x)=\frac{1}{1-L(r_1)(x)}
        \end{aligned}$$
        (czwarte przejście wynika z tego, że języki $L(r_1)^n$ i $L(r_1)^m$ są rozłączne dla $n\neq m$, a kolejne przejście z tego, że wyrażenia $r_1^n$ są jednoznaczne dla każdego $n\in\mathbb{N}$, czyli każde słowo $x\in L(r_1)^n$ ma dokładnie jeden rozkład $x=x_1\dotsm x_n$ taki, że $(x_1,\dotsc,x_n)\in \underbrace{L(r_1)\times\dotsm\times L(r_1)}_n$, a ponadto
        $$|x_1 \dotsm x_n|=|x_1|+\ldots+|x_n|=|(x_1,\dotsc,x_n)|$$
        dla $x_1,\dotsc,x_n\in L(r_1)$, więc funkcja tworząca języka $L(r_1)^n$ jest równa funkcji tworzącej języka $\underbrace{L(r_1)\times\dotsm\times L(r_1)}_n$).
\end{proof}

\subsection{Podstawowe twierdzenie o zliczaniu słów}

Przedstawimy teraz kluczowe twierdzenie w kontekście zliczania liczby słów ustalonej długości należących do języka regularnego. Twierdzenie to pochodzi z \cite{Flajolet_08}.


\begin{theorem}\label{generic_function_dfa}
    Niech $M=(Q,\Sigma,q_0,\delta,F)$ będzie skończonym automatem deterministycznym, gdzie $Q=\{q_0,\dotsc,q_{n-1}\}$. Funkcja tworząca języka akceptowanego przez $M$ wyraża się wzorem
    $$L(M)(x)=\mathbf{u}(I-xT)^{-1}\mathbf{v}~,$$
    gdzie $\mathbf{u}$ jest wektorem $(1,0,0,\dotsc,0)$, $I$ macierzą jednostkową, $T$ jest macierzą o rozmiarze $n\times n$, której elementy są dane wzorem
    $$T_{i,j}=card\{a\in\Sigma: q_j=\delta(q_i,a)\}~,$$
    a $\mathbf{v}=(v_0,\dotsc,v_{n-1})^T$ jest wektorem kolumnowym takim, że $v_i=\llbracket q_i\in F \rrbracket$.
\end{theorem}
\begin{proof}
    Niech $L_q$ oznacza język słów, które byłyby akceptowalne przez $M$, gdyby stanem początkowym był stan $q$. Dla każdego stanu $q\in Q$ zachodzi
    $$L_q=\Delta_q\cup\bigcup_{a\in\Sigma}{\left\{aw: w\in L_{\delta(q,a)}\right\}}~,$$
    gdzie $\Delta_q=\{\varepsilon\}$, jeśli $q\in F$, a w przeciwnym razie $\Delta_q=\emptyset$. Zauważmy, że dla różnych $a_1,a_2\in\Sigma$ zbiory $\left\{a_1w: w\in L_{\delta(q,a_1)}\right\}$, $\left\{a_2w: w\in L_{\delta(q,a_2)}\right\}$ i $\Delta_q$ są parami rozłączne. Ponadto klasy $(\left\{aw: w\in L_{\delta(q,a)}\right\},|\cdot|)$ i $(\{a\}\times L_{\delta(q,a)},|\cdot|)$ są izomorficzne. Wobec tego
    $$L_q\cong\Delta_q+\sum_{a\in\Sigma}{\{a\}\times L_{\delta(q,a)}}~.$$
    Przekładając to na funkcje tworzące otrzymujemy
    $$L_q(x)=\llbracket q\in F\rrbracket+x\sum_{a\in\Sigma}{L_{\delta(q,a)}(x)}=\llbracket q\in F\rrbracket+x\sum_{p\in Q}{card\{a\in\Sigma:p=\delta(q,a)\}L_p(x)}~.$$
    Po zapisaniu tego równania w postaci
    $$L_{q_i}(x)=\llbracket q_i\in F\rrbracket+x\sum_{j=0}^{n-1}{card\{a\in\Sigma:q_j=\delta(q_i,a)\}L_{q_j}(x)}=v_i+x\sum_{j=0}^{n-1}{T_{i,j}L_{q_j}(x)}$$
    dla $i=1,\dotsc,n-1$ i podstawieniu $\mathbf{L}(x)=(L_{q_1}(x),\dotsc,L_{q_{n-1}}(x))^T$ otrzymujemy
    $$\mathbf{L}(x)=\mathbf{v}+xT\,\mathbf{L}(x),$$
    co po przekształceniach daje $\mathbf{L}(x)=(I-xT)^{-1}\mathbf{v}$. Interesuje nas funkcja tworząca języka $L(M)=L_{q_0}$, zatem finalnie
    $$L(M)(x)=L_{q_0}(x)=\mathbf{u}\cdot\mathbf{L}(x)=\mathbf{u}(I-xT)^{-1}\mathbf{v}~.$$
\end{proof}

Zauważmy, że przy obliczaniu funkcji tworzącej języka akceptowanego przez DFA wykonujemy tylko podstawowe operacje elementarne. Konsekwencją tej obserwacji jest poniższe twierdzenie.

\begin{theorem}
    Funkcja tworząca dowolnego języka regularnego jest funkcją wymierną, to jest funkcją dającą się przedstawić w postaci $\frac{P(x)}{Q(x)}$, gdzie $P(x)$ i $Q(x)$ są wielomianami zmiennej $x$. Ponadto, jeśli $0$ nie jest pierwiastkiem wielomianu $P(x)$, to nie jest też pierwiastkiem wielomianu $Q(x)$.
\end{theorem}
\begin{proof}
    Z każdym językiem regularnym związany jest pewien deterministyczny automat skończony $M$. Funkcję tworzącą tego języka możemy obliczyć ze wzoru $$L(M)(x)=\mathbf{u}(I-xT)\mathbf{v}$$ przy oznaczeniach jak w twierdzeniu \ref{generic_function_dfa}. Wszystkie elementy wektorów $\mathbf{u}$, $\mathbf{v}$ oraz macierzy $I$ i $T$ możemy traktować jako funkcje wymierne. Podczas wyznaczania macierzy odwrotnej oraz obliczania iloczynu dwóch macierzy wykonujemy wyłącznie operacje dodawania, odejmowania, mnożenia i dzielenia na elementach macierzy. Klasa funkcji wymiernych jest zamknięta na te operacje, zatem wynikiem tych działań jest funkcja wymierna.\\
    Przyjrzyjmy się teraz drugiej części tego twierdzenia. Niech $L$ będzie językiem regularnym i $L(x)=\frac{P(x)}{Q(x)}$. Gdyby $0$ było pierwiastkiem wielomianu $Q(x)$, ale nie byłoby pierwiastkiem wielomianu $P(x)$, to wartość $L(0)$ nie byłaby określona, jednak $L(0)=\sum_{k=0}^{\infty}{l_n0^n}=0$. Wobec tego, jeśli wielomiany $P(x)$ i $Q(x)$ są względnie pierwsze (nie istnieje wielomian stopnia większego od $0$ dzielący oba te wielomiany), to $0$ nie jest pierwiastkiem wielomianu $Q(x)$.
\end{proof}

\begin{theorem}
    Każda funkcja wymierna $\frac{P(x)}{Q(x)}$ posiada rozkład na ułamki proste
    $$\frac{P(x)}{Q(x)}=R(x)+\sum_{i=1}^{m}{\sum_{j=1}^{a_i}{\frac{A_{i,j}}{(x_i-x)^{j}}}}~,$$
    gdzie $Q(x)$ ma $m$ różnych pierwiastków zespolonych $x_i$ o krotności $a_i$, $A_{i,j}$ są stałymi, a $R(x)$ jest wielomianem stopnia $\deg{P(x)}-\deg{Q(x)}$, jeśli $\deg{P(x)}\geq\deg{Q(x)}$, a zerem w przeciwnym razie.
\end{theorem}

Skorzystajmy z powyższych twierdzeń do wyznaczenia zwartego wzoru na funkcję tworzącą języka regularnego. Niech $L$ będzie językiem regularnym oraz
$$L(x)=\frac{P(x)}{Q(x)}=R(x)+\sum_{i=1}^{m}{\sum_{j=1}^{a_i}{\frac{B_{i,j}}{(1-b_{i,j}x)^{j}}}}$$
(taką postać $L(x)$ otrzymujemy po podstawieniu $b_{i,j}=x_i^{-j}$ i $B_{i,j}=A_{i,j}b_{i,j}$ we wzorze z powyższego twierdzenia, możemy założyć, że $0$ nie jest pierwiastkiem wielomianu $Q(x)$). Wówczas, korzystając ze wzoru (\ref{eq:genBinomial}), otrzymujemy, że liczba słów długości $n$ należących do tego języka wynosi
\begin{equation} \label{eq:basicRational}
    \begin{aligned}
        l_n&=[x^n]L(x)=[x^n]R(x)+\sum_{i=1}^{m}{\sum_{j=1}^{a_i}{[x^n]\frac{B_{i,j}}{(1-b_{i,j}x)^j}}}\\
        &=[x^n]R(x)+\sum_{i=1}^{m}{\sum_{j=1}^{a_i}{\binom{n+j-1}{n}B_{i,j}b_{i,j}^n}}~.
    \end{aligned}
\end{equation}
W dalszych rozdziałach wykorzystamy powyższe obserwacje do zliczania słów ustalonej długości należących do języków regularnych generowanych przez wybrane wyrażenia regularne.

We wzorze (\ref{eq:basicRational}) może się zdarzyć, że liczba $m$ pierwiastków wielomianu $Q(x)$ jest równa zeru. Wtedy funkcja $L$ jest wielomianem, więc ciąg liczb $[x^n]L(x)$ ma wielomianowe tempo wzrostu. W przeciwnym przypadku, gdy $m>0$ we wzorze  (\ref{eq:basicRational}) pojawiają się czynniki postaci 
$$\frac{A}{(c- x)^k}~.$$
Ze uogólnionego wzoru dwumianowego Newtona wynika, że odpowiadają one za wykładniczy wzrost ciągu liczb $[x^n]L(x)$. 
Z powyższych rozważań wynika następująca ciekawa obserwacja:
\begin{theorem}
Niech  $R$ będzie językiem regularnym nad skończonym alfabetem $\Sigma$. Niech
$$
r_n = |R\cap\Sigma^n|~. 
$$
Wtedy możliwe są tylko dwa przypadki:
\begin{enumerate}
    \item istnieje wielomian $w(x)$ taki, że $r_n \leq w(n)$ dla każdego $n$ 
    \item istnieje stała $c>0$ taka, że $r_n \geq c^n$ dla każdego $n$
\end{enumerate}
\end{theorem}