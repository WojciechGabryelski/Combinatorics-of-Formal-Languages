% !TEX encoding = UTF-8 Unicode 
% !TEX root = praca.tex

\chapter{Podsumowanie}

Zaimplementowane narzędzia omówione w tej pracy pozwalają efektywnie wyznaczyć funkcję tworzącą języka generowanego przez wyrażenie regularne. Mając daną tę funkcję, przy użyciu różnych pakietów matematycznych (np. Mathematica, Wolfram Alpha, Maxima, Matlab, Scilab), można następnie wyznaczyć wzór na liczbę słów długości $n$ należących do tego rozpatrywanego języka, który następnie można zastosować dla wielu wartości $n$. 

Powyższe narzędzia mogą być szczególnie przydatne przy badaniu wpływu ograniczeń narzuconych na hasła przez wyrażenie regularne na podatność na złamanie systemu bezpieczeństwa. Innym potencjalnym zastosowaniem tego narzędzia może być wyznaczanie minimalnej liczby bitów, na których można zapisać słowa ograniczonej długości należące do języka opisanego przez dane wyrażenie regularne, a następnie próba kompresji tych słów. Zagadnienie to nie było rozważane w tej pracy.

Zrealizowany w ramach tej pracy program można rozszerzyć o funkcje wyznaczanie pierwiastków mianownika otrzymanej funkcji tworzącej, a następnie zamienienie tej funkcji na sumę ułamków prostych, dzięki czemu byłoby możliwe automatyczne (bez użycia innych systemów) wyznaczenie współczynników w rozwinięciu tej funkcji w szereg potęgowy, a tym samym wyznaczenie liczby słów o zadanej długości należących do rozważanego języka regularnego. Inną kwestią, którą można poruszyć w dalszych rozważaniach, jest zbadanie złożoności obliczeniowej zaimplementowanych algorytmów oraz podjęcie próby ich zoptymalizowania.

W trakcie badań przeprowadzonych w trakcie realizacji pracy natrafiliśmy na ciekawy, naszym zdaniem, problem: 
\begin{quote}
Załóżmy, że znamy funkcje tworzące $R_1(x)$ i $R_2(x)$ dwóch języków regularnych $L_1$ oraz $L_2$; czy istnieje nietrywialna zależność algebraiczna między $R_1(x)$, $R_2(x)$ oraz funkcją tworzącą przekroju $L_1 \cap L_2$?
\end{quote}  
Zauważmy, że produkt dwóch języków regularnych jest językiem regularnym. Operacja przekroju wyrażeń nie jest traktowana jako operacja elementarna na wyrażeniach regularnych. Jednak jest ona pożyteczna, jest łatwa do implementacji w algorytmach służących do rozpoznawania zgodności łańcucha ze wzorcem oraz konstrukcja automatu rozpoznającego  $L_1 \cap L_2$ z automatów rozpoznających $L_1$ i $L_2$ jest bardzo prosta.