% !TEX encoding = UTF-8 Unicode 
% !TEX root = praca.tex

\chapter{Pojęcia kombinatoryczne}

W rozdziale tym omówimy pojęcia kombinatoryczne, które wykorzystywać będziemy w pracy. Omówione tutaj własności uogólnionych współczynników dwumianowych znaleźć można w klasycznej książce \cite{knuth1989concrete}, zaś pojęcie, podstawowe konstrukcje i podstawowe własności klas kombinatorycznych znaleźć można w książce \cite{Flajolet_08}.

\section{Oznaczenia}
W pracy będziemy stosowali standardową notację matematyczną. W szczególności korzystać będziemy z następujących oznaczeń:
\begin{enumerate}
    \item $\mathbb{N}$ - zbiór liczb naturalnych (do zbioru tego zaliczamy również liczbę $0$)
    \item $\mathbb{C}$ - zbiór liczb zespolonych 
    \item $[n]$ - zbiór liczb $\{1,\dotsc,n\}$ dla $n \in \mathbb{N}$ $([0] = \emptyset)$
    \item $\llbracket p \rrbracket = \left\{
        \begin{array}{rl}
            1, & \text{zdanie $p$ jest prawdziwe}\\
            0, & \text{zdanie $p$ jest fałszywe}
        \end{array}
        \right.$
    \item $[x^n]A(x)$ - współczynnik przy $x^n$ w szeregu $A(x) = \sum\limits_{n=0}^{\infty} a_n x^n$
\end{enumerate}

\section{Elementy kombinatoryki}

Klasyczny symbol dwumianowy $\binom{n}{k}$ określony jest dla liczb naturalnych $n$ oraz $k$. Można go uogólnić do funkcji $\binom{x}{k}$ dla dowolnych $x \in \mathbb{C}$:
$$
\binom{x}{k} =  \frac{\prod_{a=0}^{k-1}(x-a)}{k!}
$$
Dla tak określonego uogólnienia współczynników dwumianowych zachodzi następujące uogólnienie wzoru dwumianowego Newtona
$$
(1+x)^{\alpha} = \sum_{k\geq 0} \binom{\alpha}{k} x^k~,\quad |x|<1~.
$$ 
Korzystając z następującej tożsamości (zwanej "górną negacją"): 
$\binom{x}{k} = (-1)^k \binom{k-x-1}{k}$ otrzymujemy następujący ważny dla nas wzór:
\begin{equation}\label{eq:genBinomial}
\frac{1}{(1-x)^n} = \sum_{k\geq 0} \binom{n+k-1}{k} x^n, \quad |x|<1~.
\end{equation}

Liczbami Fibonacciego nazywamy liczby $F_n$ zdefiniowane w następujący sposób: $F_0=0$, $F_1=1$ oraz $F_n = F_{n-1}+F_{n-2}$ dla $n\geq 2$. Funkcja tworząca ciągu liczb Fibonacciego wyraża się wzorem
\begin{equation} \label{eq:Fib}
\sum\limits_{n=0}^{\infty} F_n x^n = \frac{x}{1-x-x^2}~.
\end{equation}

\section{Klasy kombinatoryczne}

\begin{definition}
    \emph{Klasą kombinatoryczną} nazywamy parę $\mathcal{C}=(C,|\cdot|)$ taką, że $C$ jest zbiorem niepustym, a $|\cdot|:C \to \mathbb{N}$ jest funkcją ze zbioru $C$ w zbiór liczb naturalnych taką, że dla każdego $n\in \mathbb{N}$ zbiór $\{c\in C: |c|=n\}$ jest skończony. Liczbę $|c|$ nazywamy wagą lub rozmiarem elementu $c$.
\end{definition}

Klasę kombinatoryczną będziemy oznaczali kręconą literą, natomiast zbiór tej klasy będziemy oznaczali prostą literą. Dla klasy kombinatorycznej $\mathcal{C} = (C,|\cdot|)$  definiujmy
\begin{enumerate}
    \item $C_n = \{c\in C: |c|=n\}$
    \item $c_n = card(C_n)$
\end{enumerate}
Analogiczne oznaczenia będziemy stosować dla innych liter alfabetu.

\begin{definition}
    \emph{Funkcją tworzącą} klasy $\mathcal{C}=(C,|\cdot|)$ będziemy nazywać (formalny) szereg potęgowy
    $$\mathcal{C}(x) = \sum_{n=0}^{\infty}{c_nx^n}~.$$
\end{definition}

Zauważmy, że
\begin{equation*}
    \begin{aligned}
        \mathcal{C}(x) &= \sum_{n=0}^{\infty}{c_nx^n} = \sum_{n=0}^{\infty}{card(\{c\in C: |c|=n\})x^n} = \sum_{n=0}^{\infty}{x^n\sum_{c\in C}{\llbracket |c| = n \rrbracket}}\\
        &= \sum_{c\in C}{\sum_{n=0}^{\infty}{\llbracket |c| = n \rrbracket x^n}} = \sum_{c\in C}{x^{|c|}}.
    \end{aligned}
\end{equation*}
W naszych rozważaniach często będziemy mieli do czynienia ze zwartą postacią szeregu $\mathcal{C}(x)$. Wówczas przyda się oznaczenie $[x^n]\mathcal{C}(x)=c_n$.

\begin{definition}
    Dwie klasy $\mathcal{A}$ i $\mathcal{B}$ nazywamy \emph{izomorficznymi} ($\mathcal{A}\cong\mathcal{B}$), gdy ich funkcje tworzące są identyczne.
\end{definition}

Łatwo można zauważyć, że klasy kombinatoryczne $\mathcal{A}=(A,|\cdot|_{\mathcal{A}})$ oraz $\mathcal{B}=(B,|\cdot|_{\mathcal{B}})$ są izomorficzne wtedy i tylko wtedy, gdy istnieje bijekcja $f:A\to B$ taka, że dla każdego $x\in A$ mamy $|f(x)|_{\mathcal{B}} = |x|_{\mathcal{A}}$.


\subsection{Podstawowe konstrukcje klas kombinatorycznych}

W części tej omówimy podstawowe metody konstruowania klas kombinatorycznych oraz kilka bazowych klas. 
Niech 
$$\mathcal{E}=(\{\varepsilon\},|\cdot|_{\mathcal{E}})~,$$ 
gdzie $|\varepsilon|_{\mathcal{E}}=0$ oraz 
$$\mathcal{X}=(\{\bullet\},|\cdot|_{\mathcal{X}})~,$$ 
gdzie $|\bullet|_{\mathcal{X}}=1$. Funkcje tworzące tych klas to $\mathcal{E}(x)=1$ oraz $\mathcal{X}(x)=x$. 

Przedstawimy teraz trzy podstawowe konstrukcje klas kombinatorycznych.
Niech $\mathcal{A}=(A,|\cdot|_{\mathcal{A}})$ i $\mathcal{B}=(B,|\cdot|_{\mathcal{B}})$ będą klasami kombinatorycznymi. 
\begin{enumerate}
    \item Sumą klas $\mathcal{A}$ i $\mathcal{B}$ nazywamy klasę
    $$\mathcal{A}+\mathcal{B} = \left((A\times\{\bullet\})\cup (B\times\{\circ\}), |\cdot|_{\mathcal{A}+\mathcal{B}}\right),$$
    gdzie $\bullet$ oraz $\circ$ są dowolnymi różnymi elementami (suma klas kombinatorycznych $\mathcal{A}$ i $\mathcal{B}$ wymaga urozłącznienia zbiorów $A$ i $B$), a funkcja $|\cdot|_{\mathcal{A}+\mathcal{B}}:(A\times\{\bullet\})\cup (B\times\{\circ\}) \to \mathbb{N}$ wyraża się wzorem
    $$|c|_{\mathcal{A}+\mathcal{B}} = \left\{
    \begin{array}{rl}
        |a|_{\mathcal{A}}, & c=(a,\bullet) \text{ dla  } a \in A\\
        |b|_{\mathcal{B}}, & c=(b,\circ) \text{ dla } b \in B
    \end{array}
    \right.$$
    
    \item Produktem klas $\mathcal{A}$ i $\mathcal{B}$ nazywamy klasę 
    $$\mathcal{A}\times\mathcal{B} = (A\times B, |\cdot|_{\mathcal{A}\times\mathcal{B}}),$$
    gdzie funkcja $|\cdot|_{\mathcal{A}\times\mathcal{B}}:A\times B \to \mathbb{N}$ wyraża się wzorem
    $$|(a,b)|_{\mathcal{A}\times\mathcal{B}} = |a|_{\mathcal{A}}+|b|_{\mathcal{B}}$$
    
    \item Ciągami elementów klasy $\mathcal{A}$ takiej, że $a_0 = 0$ nazywamy klasę 
    $$\SEQ{A} = \mathcal{E}+\mathcal{A}+(\mathcal{A}\times\mathcal{A})+(\mathcal{A}\times\mathcal{A}\times\mathcal{A})+\dots$$
\end{enumerate}

Operacja $\mathbf{SEQ}$ jest określona tylko dla takich klas kombinatorycznych $\mathcal{A}$, że $a_0 = 0$. Gdyby w klasie $\mathcal{A}$ występował element $e$ taki, że $|e|_{\mathcal{A}} = 0$, to do sumy 
$\bigcup_{n\geq 0} A^n$ należałyby elementy $(e)$, $(e,e)$, $(e,e,e)$, ... więc w sumie tej występowałoby nieskończenie wiele elementów rozmiaru $0$.

\begin{theorem} Dla dowolnych klas kombinatorycznych $\mathcal{A}$ oraz $\mathcal{B}$ mamy
\begin{enumerate}
\item $(\mathcal{A}+\mathcal{B})(x) = \mathcal{A}(x)+\mathcal{B}(x)$
\item $(\mathcal{A}\times\mathcal{B})(x) = \mathcal{A}(x)\cdot\mathcal{B}(x)$
\item $(\SEQ{A})(x) = (1-\mathcal{A}(x))^{-1}$, o ile $a_0 = 0$
\end{enumerate}
\end{theorem}

Operacje arytmetyczne (dodawanie, mnożenie, odwrotność) występujące w powyższym twierdzeniu są operacjami na szeregach formalnych. W szczególności, produkt jest interpretowany jako iloczyn Cauchy'ego szeregów. Jeśli rozważane szeregi mają niezerowe promienie zbieżności, to mogą być interpretowane jako szeregi potęgowe i wtedy owe operacje formalne pokrywają się z analitycznymi operacjami na szeregach potęgowych. W szczególności, ostatni wzór można zapisać w postaci
$$
(\SEQ{A})(x) = \frac{1}{1-\mathcal{A}(x)}
$$
\begin{proof}
(1) Pierwsza część twierdzenia wynika z następującego ciągu równości :
$$
(\mathcal{A}+\mathcal{B})(x) = 
    \sum_{c\in (A\times\{\bullet\})\cup (B\times\{\circ\})}{x^{|c|_{\mathcal{A}+\mathcal{B}}}} = 
    \sum_{a \in A}{x^{|a|_{\mathcal{A}}}} + \sum_{b \in B}{x^{|b|_{\mathcal{B}}}} = 
    \mathcal{A}(x)+\mathcal{B}(x) ~.
$$
(2) Druga część twierdzenia wynika z następujących równości:
\begin{gather*}
(\mathcal{A}\times \mathcal{B})(x) = 
\sum_{c\in A\times B}{x^{|c|_{\mathcal{A}\times\mathcal{B}}}} = 
\sum_{(a,b)\in A\times B}{x^{|a|_{\mathcal{A}}+|b|_{\mathcal{B}}}} = 
\sum_{(a,b)\in A\times B}{x^{|a|_{\mathcal{A}}}x^{|b|_{\mathcal{B}}}} =\\
    \left(\sum_{a\in A}{x^{|a|_{\mathcal{A}}}}\right)\left(\sum_{b\in B}{x^{|b|_{\mathcal{B}}}}\right) = 
\mathcal{A}(x)\cdot\mathcal{B}(x)~.
\end{gather*}
%
(3) Zauważmy, że
$$
(\SEQ{A})(x) = 1+\mathcal{A}+(\mathcal{A}\times\mathcal{A})(x)+(\mathcal{A}\times\mathcal{A}\times\mathcal{A}))(x) + \ldots = 
\sum_{n\geq 0} \mathcal{A}(x)^n~.
$$
Następnie
\begin{gather*}
\left(\sum_{n\geq 0} \mathcal{A}(x)^n\right) (1-\mathcal{A}(x)) = 
\left(\sum_{n\geq 0} \mathcal{A}(x)^n\right) - 
\left(\sum_{n\geq 0} \mathcal{A}(x)^{n+1}\right) = 1~,
\end{gather*}
więc szereg $1-\mathcal{A}(x)$ jest formalną odwrotnością 
szeregu $(\SEQ{A})(x)$ w pierścieniu szeregów formalnych.

\end{proof}

W książce \cite{Flajolet_08} znaleźć można wiele innych konstrukcji klas kombinatorycznych (np. operacje zbioru potęgowego, operację cykli). Jednak w tej pracy nie będziemy z nich korzystać, więc nie omawiamy ich.

Warto zaznaczyć, że pojęcie klasy kombinatorycznych należy do dziedziny zwanej kombinatoryką symboliczną. Umożliwia ona wyznaczanie funkcji tworzących opisujących rozważane obiekty kombinatoryczne za pomocą metod algebraicznych.  

W dalszych częściach pracy stosować będziemy klasy kombinatoryczne  do badania własności języków regularnych. W takich przypadkach w rozważanych przez nas przypadkach dla ustalonego skończonego alfabetu $\Sigma = \{a_1,\ldots,a_n\}$ bazową klasą kombinatoryczną będzie klasa
$\mathcal{A} = (\{a_1,\ldots,a_n\}, |\cdot|\}$ dla której określamy $|a_n| = 1$, dla wszystkich $i\in [n]$. Inaczej mówiąc, ustalamy, że wszystkie elementy języka mają tę samą wagę równą $1$. Zauważmy, że dla takiej klasy $\mathcal{A}$ mamy 
$$\mathcal{A}(x) = n \cdot x$$
oraz
$$\SEQ{A}(x) = \frac{1}{1-n \cdot x}~.$$