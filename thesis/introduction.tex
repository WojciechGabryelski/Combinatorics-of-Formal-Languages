\chapter*{Wprowadzenie}

Praca dyplomowa poświęcona jest problemowi zliczania słów języka regularnego o ustalonym rozmiarze. Zauważyć bowiem można, że pewne klasy języków regularnych są łatwe do opisania (np. wyrażanie regularne $(a+b)^*aa(a+b)^*$ opisuje słowa w których występują pod rząd kolejne dwa wystąpienia litery $a$), jednak bezpośrednie tłumaczenie takiego wyrażenia na funkcje tworzące nie daje prawidłowej odpowiedzi. W pracy omówimy metodę dokładnego wyznaczania liczby słów ustalonej długości dla zadanego wyrażenia regularnego.

W rozdziale 1 wprowadzamy wykorzystywane w dalszej części pracy oznaczenia, fakty i wzory. W szczególności omawiamy pojęcie klasy kombinatorycznej oraz związanych z nimi pojęcie funkcje tworzące. Podamy również uogólniony wzór dwumianowy Newtona.

W rozdziale 2 przedstawiamy pojęcia wyrażeń regularnych, języków regularnych i automatów skończonych oraz omawiamy związki między nimi. 

W rozdziale 3 przedstawiamy metodę wyznaczania funkcji tworzącej (wprowadzonej w rozdziale 1) dla danego języka regularnego, a następnie obliczania dokładnej liczby słów zadanej długości należących do tego języka.

W rozdziale 4 omawiamy działanie programu wyznaczającego funkcję tworzącą języka generowanego przez podane wyrażenie regularne oraz role poszczególnych klas i metod w całym tym procesie.

W rozdziale 5 prezentujemy funkcje tworzące kilku języków regularnych danych przez wybrane wyrażenia regularne oraz przeprowadzamy badania wpływu ograniczeń narzuconych przez wyrażenia regularne na liczbę haseł zadanej długości opisanych przez to wyrażenie przy wykorzystaniu zaimplementowanego oprogramowania.

Rozdział 6 zawiera krótkie podsumowanie pracy dyplomowej, możliwości rozbudowania zbudowanych narzędzi oraz możliwości ich wykorzystania do innych zagadnień. 

Wszystkie zrealizowane i omawiane w pracy algorytmy zostały napisane w wersji 11 języka C++ (patrz np. \cite{meyers2014effective}) i skompilowane na urządzeniu z systemem operacyjnym Windows 10 oraz na maszynie wirtualnej z systemem Linux (wymagana jest instalacja biblioteki GMP do obliczeń numerycznych dowolnej dokładności (patrz \cite{10.5555/2911024}), więc rekomendowanym systemem operacyjnym do testowania programu jest system Linux ze względu na łatwość instalacji tej biblioteki).