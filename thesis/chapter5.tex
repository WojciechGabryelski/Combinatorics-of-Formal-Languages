\chapter{Badania wybranych wyrażeń regularnych} \label{chapt:wybrane}

W rozdziale tym wykorzystamy zaimplementowane narzędzia do wyznaczenia funkcji tworzących języków regularnych zadanych przez wybrane wyrażenia regularne.

\section{Klasyczne przykłady}

\subsection{Słowa zawierające "aa"}
Wróćmy na chwilę do przykładu z rozdziału 3. Po wprowadzeniu przez użytkownika wyrażenia regularnego $r=(a+b)^*aa(a+b)^*$ program zwraca funkcję tworzącą $\frac{x^2}{(1-2x)(1-x-x^2)}$. 
Zauważmy, że 
$$
\frac{x^2}{(1-2x)(1-x-x^2)} =
\frac{1}{1-2x} - 
\frac{x+1}{1-x-x^2}
$$
Przypomnijmy, że $\frac{1}{1-2 x} = \sum_{n=0}^{\infty} 2^n x^n$. Mamy również (patrz wzór (\ref{eq:Fib}))
$$
  \frac{x}{1-x-x^2} = \sum_{n=0}^{\infty} F_n x^n~,
$$
gdzie $F_n$ oznacza $n$-tą liczbę Fibonacciego. Zatem
\begin{gather*}
\frac{x+1}{1-x-x^2} = \frac{x}{1-x-x^2} + \frac{1}{1-x-x^2} = \\
\sum_{n=0}^{\infty} F_n x^n + \sum_{n=0}^{\infty} F_{n+1} x^n = \\
\sum_{n=0}^{\infty} (F_n+F_{n+1}) x^n = \sum_{n=0}^{\infty} F_{n+2} x^n
\end{gather*}
Zatem, liczba słów długości $n$ języka nad $\{a,b\}$ zawierających podciąg $aa$ wyraża się wzorem
$$
  r_n = 2^n - F_{n+2}~.
$$
Otrzymaliśmy więc wzór (\ref{eq:correctFormula}) o którym wspomnieliśmy w Rozdziale 3.

\subsection{Słowa zawierające "aa" i "bb"}

Rozważmy język słów złożonych z liter $a$ i $b$, które zawierają co najmniej jedno wystąpienie podciągu $aa$ i co najmniej jedno wystąpienie słowa $bb$. Wyrażeniem regularnym opisującym ten język jest wyrażenie
$$r_1=(a+b)^*(aa(a+b)^*bb+bb(a+b)^*aa)(a+b)^*~,$$
jednak nie jest to wyrażenie jednoznaczne. W celu znalezienia jednoznacznego wyrażenia regularnego opisującego ten język rozpatrzmy osobno słowa, w których pierwsze wystąpienie podciągu $aa$ znajduje się przed pierwszym wystąpieniem podciągu $bb$ oraz słowa, w których pierwsze wystąpienie podciągu $aa$ znajduje się za pierwszym wystąpieniem podciągu $bb$. Otrzymujemy w ten sposób wyrażenie
$$r_2=(\varepsilon+b)(ab)^*aaa^*(baa^*)^*bb(a+b)^*+(\varepsilon+a)(ba)^*bbb^*(abb^*)^*aa(a+b)^*~.$$
Można pokazać, że wyrażenie to jest jednoznaczne. Stąd funkcja tworząca tego języka wynosi
\begin{equation*}
    \begin{aligned}
        &L(r_1)(x)=L(r_2)(x)=\frac{2(1+x)x^4}{(1-x^2)(1-x)(1-2x)(1-\frac{x^2}{1-x})}\\
        &=\frac{2x^4}{(1-x)(1-2x)(1-x-x^2)}=\frac{1}{1-2x}-\frac{2(x+1)}{1-x-x^2}+\frac{2}{1-x}-1~,
    \end{aligned}
\end{equation*}
zatem liczba ciągów liter $a$ i $b$ długości $n$ zawierająca podciągi $aa$ i $bb$ wynosi
$$l_n=2^n-2F_{n+2}+2-\llbracket n=0\rrbracket~.$$
Funkcja tworząca wyznaczona przez zaimplementowany program jest identyczna.

Zauważmy, że im więcej jest podciągów, które musi zawierać rozważany język, tym dłuższe jest wyrażenie regularne opisujące ten język, musimy bowiem rozważyć wszystkie kolejności, w jakich występują pierwsze wystąpienia takich podciągów. Liczba różnych kolejności jest równa liczbie permutacji zbioru złożonego z tych podciągów, zatem wyrażenie regularne rośnie bardzo szybko wraz ze zwiększaniem liczby podciągów, co powoduje duży wzrost liczby stanów skończonego automatu deterministycznego związanego z tym wyrażeniem regularnym oraz znaczne zwiększenie czasu oczekiwania wynik zwrócony przez zaimplementowany program. Wówczas lepiej jest zastanowić się nad wyrażeniem jednoznacznym opisującym taki język i przetłumaczyć to wyrażenie bezpośrednio na funkcję tworzącą.

\subsection{Słowa o ustalonym znaku na danej pozycji}

Rozważmy wyrażenie regularne
$$r_k=(a+b)^*a(a+b)^{k-1}~,$$
które opisuje wszystkie ciągi liter $a$ i $b$, w których $k$-ty symbol od końca jest literą $a$. Jest to znany przykład wyrażeń, dla których klasyczna konstrukcja automatu skończonego zawiera tzw. eksplozję wykładniczą. Jednakże problem zliczania dla tego języka jest bardzo prosty, gdyż wystarczy zauważyć, że liczba słów tego języka długości $n$ jest równa liczbie słów długości $n$ języka generowanego przez wyrażenie
$$s_k=(a+b)^{k-1}a(a+b)^*~,$$
która jest równa $0$ dla $n<k$ oraz $2^{n-1}$ dla $n\geq k$. Dla $k=11$ NFA bez $\varepsilon$-przejść, wygenerowany za pomocą zaimplementowanego programu, dla pierwszego wyrażenia zawiera 36 stanów, a dla drugiego 32 stany, natomiast DFA otrzymany przez przekształcenie NFA w DFA dla pierwszego wyrażenia zawiera aż 2049 stanów (po minimalizacji 2048), natomiast dla drugiego wyrażenia ma ich zaledwie 12. Wniosek, który możemy wysnuć z tych obserwacji, jest taki, że w przypadku, gdy chcemy wyznaczyć liczbę słów z języka danego przez wyrażenie regularne, a DFA wygenerowany na podstawie tego wyrażenia ma dużą liczbę stanów, warto zastanowić się nad innym izomorficznym językiem, w przypadku którego DFA akceptujący ten język może zawierać znacznie mniejszą liczbę stanów.

Warto również odnotować, że funkcja tworząca języka opisanego przez to wyrażenie regularne jest równa $f(x)=\frac{2^{k-1}x^k}{1-2x}$, a zapisana w postaci kanonicznej $f(x)=R(x)+\frac{P(x)}{Q(x)}$, gdzie $\deg{P(x)}<\deg{Q(x)}$, jest dużo mniej czytelna. Stąd warto czasem pozostawić funkcję tworzącą w postaci $\frac{P(x)}{Q(x)}$ w celu jaśniejszej interpretacji.

\section{Siła haseł}

Hasła są standardowym rozwiązaniem służącym do ograniczania dostępu do systemów informatycznych do uprawnionych użytkowników. Z powodu powszechności ich występowania i obecności wielu systemów użytkownicy serwisów mają tendencję do stosowania prostych do przewidzenia haseł, co ułatwia stosowanie ataków na takie zabezpieczenia za pomocą metod słownikowych lub technik typu "brute force" (systematycznego przeszukiwania wszystkich możliwych haseł). Jedną z metod stosowanych przez projektantów systemów informatycznych do uniemożliwiania stosowania stosunkowo prostych do złamania prostych haseł polega na stawianiu różnego rodzaju ograniczeń na możliwe do stosowania hasła.    

W rozdziale tym zajmiemy się  zbadaniem jednego, często stosowanego, typu ograniczeń. Załóżmy, że nasze hasła mogą się składać z dowolnych małych i wielkich liter alfabetu łacińskiego (razem z 52 liter), 10 cyfr oraz 10 powszechnie stosowanych znaków specjalnych. Łącznie nasze hasła mogą zawierać 72 różne znaki. 
Nałóżmy na możliwe do zastosowania hasła następujące ograniczenie:
\begin{quote}
    W haśle występować ma przynajmniej jedna duża litera, po której występuje  przynajmniej jedna mała litera, po której występuje przynajmniej jedna cyfra, po której wystąpić ma przynajmniej jeden znak specjalny. 
\end{quote}
Jest do dosyć silne ograniczenie, gdyż oprócz wymuszenia stosowania dużych liter, małych liter, cyfr i znaków specjalnych narzucona jest w nim kolejność występowania.

Niech $\Sigma$ oznacza zbiór wszystkich rozważanych znaków. Mamy $|\Sigma| = 72$. 
Niech $r_1$, $r_2$, $r_3$ i $r_4$ będą wyrażeniami regularnymi przedstawionymi w postaci sumy odpowiednio wszystkich małych liter, wielkich liter, cyfr i rozważanych znaków specjalnych (na przykład $r_1=a+b+\ldots+z$). 
Następujące wyrażenie regularne jest bezpośrednim tłumaczeniem rozważanego ograniczenia:
$$
 r = \Sigma^*r_1\Sigma^*r_2\Sigma^*r_3\Sigma^*r_4\Sigma^*
$$
Po wykorzystaniu zbudowanego w pracy narzędzia  wyrażenie to generuje następującą funkcję tworzącą
$$
f(z) = \frac{67600z^4}{(1-72z)(1-62z)^2(1-46z)^2}~.
$$
Oto rozkład otrzymanej funkcji na ułamki proste:
\begin{gather*}
f(z) = -\frac{325}{1472 (1-46z)^2}+\frac{8619}{7936 (1-62z)}-\frac{845}{1984 (1-62z)^2}\\+\frac{1}{1-72z}-\frac{8475}{5888 (1-46z)}~.
\end{gather*}
Otrzymany rozkład umożliwia otrzymanie za pomocą wzoru (\ref{eq:genBinomial}) zwartych formuł na liczbę słów rozważanego języka ustalonej długości. Po serii żmudnych, acz elementarnych, przekształceń otrzymujemy następujący wzór:
$$
r_n = 72^n-\frac{1}{182528} (62^n (77740 n-120497)+46^n (40300 n+303025))
$$
Liczba $r_n - 72^n$ jest liczbą wszystkich ciągów rozważanego języka o długości $n$ które są niedopuszczalne przez wprowadzone ograniczenia.
Z powyższego wzoru tego wynika, że 
$$
  72^n - r_n \approx \frac{845}{1984}\cdot n \cdot 62^n \approx 0.5 \cdot n \cdot 62^n~.
$$ 

Poniższa tabela zawiera wartości liczb $r_n$ dla $n=1,\ldots,8$ oraz przybliżone wartości liczb $72^n$. 
\begin{center}
\begin{tabular}{|c|c|c|}
 \hline
 n & $r_n$ & $72^n$ \\ \hline  
 1 & $0$ & $72$ \\
 2 & $0$ & $5184$ \\
 3 & $0$ & $373248$ \\
 4 & $67600$ & $2.68739\times 10^7$ \\
 5 & $1.94688\times 10^7$    & $1.93492\times 10^9$ \\
 6 & $3.38162\times 10^9$    & $1.39314\times 10^{11}$ \\
 7 & $4.59172\times 10^{11}$ & $1.00306\times 10^{13}$ \\
 8 & $5.37093\times 10^{13}$ & $7.22204\times 10^{14}$ \\ \hline
\end{tabular}
\end{center}
Z powyższej tabeli wynika, że \textbf{bez wprowadzenia dodatkowego ograniczenia na długość stosowanych haseł w rozważanym scenariuszu dojść może do wyraźnego osłabienia bezpieczeństwa systemu}. Mianowicie, jeśli pozwolimy na wprowadzanie haseł długości 4 (a można się spodziewać, że użytkownicy będą przejawiali tendencję do stosowania krótkich haseł), to liczba dopuszczalnych przez system haseł zredukowana będzie do około 70 000.

Z powyższych rozważań wynika następujące zalecenie dla projektantów systemu wprowadzających ograniczenia na dopuszczalne hasła: 
\begin{quote}
  \emph{Jeśli narzucasz silne ograniczenia na strukturę dopuszczalnych haseł, to wśród tych ograniczeń powinno być również ograniczenie na minimalną długość dopuszczalnego hasła.}   
\end{quote}  
Jednym z narzędzi służących do analizy wpływu długości dopuszczalnych haseł na bezpieczeństwo implementowanego systemu jest system zbudowany w ramach tej pracy dyplomowej.
